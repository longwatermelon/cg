\documentclass[12pt]{article}
\usepackage{amsmath}
\usepackage{graphicx}
\usepackage{geometry}
\usepackage{indentfirst}
\usepackage{amsfonts}
\geometry{legalpaper, portrait, margin=0.5in}
\usepackage{color}   %May be necessary if you want to color links
\usepackage{hyperref}
\hypersetup{
    colorlinks=true, %set true if you want colored links
    linktoc=all,     %set to all if you want both sections and subsections linked
    linkcolor=black,  %choose some color if you want links to stand out
}
\usepackage[siunitx]{circuitikz}
\usetikzlibrary{patterns}
\usetikzlibrary{decorations.markings}
\graphicspath{ {./images/} }
\usepackage{enumitem}
\begin{document}
\newcommand*\dif{\mathop{}\!\mathrm{d}}

\setlist[enumerate]{noitemsep, topsep=0pt, leftmargin=2.5\parindent}
\setlist[itemize]{noitemsep, topsep=0pt, leftmargin=\parindent}

\newenvironment{myitemize}
{ \begin{itemize}
    \setlength{\itemsep}{0pt}
    \setlength{\parskip}{0pt}
    \setlength{\topsep}{0pt}
    \setlength{\parsep}{0pt}     }
{ \end{itemize}                  } 

\newenvironment{myenumerate}
{ \begin{enumerate}
    \setlength{\itemsep}{0pt}
    \setlength{\parskip}{0pt}
    \setlength{\topsep}{0pt}
    \setlength{\parsep}{0pt}     }
{ \end{enumerate}                  } 

\newenvironment{nopagebr}
  {\par\nobreak\vfil\penalty0\vfilneg
   \vtop\bgroup}
  {\par\xdef\tpd{\the\prevdepth}\egroup
   \prevdepth=\tpd}

\tableofcontents

\addcontentsline{toc}{section}{Table of contents}
 
\pagebreak

\section{Modeling}

\subsection{Splines}

\subsubsection{Cubic Hermite Interpolation}

Each point is defined by its position $h_n$ and slope $h_{m + n}$, $m$ being the number of control points. To simplify calculations,
it is assumed that $t_0 = 0$ and $t_1 = 1$.

The goal is to convert from a monomial basis
\begin{align*}
    \phi_0(t) &= 1\\
    \phi_1(t) &= t\\
    \phi_2(t) &= t^2\\
    \phi_3(t) &= t^3
\end{align*}

to a hermite basis
\begin{align*}
    H_0(t) &= 2t^3 - 3t^2 + 1\\
    H_1(t) &= -2t^3 + 3t^2\\
    H_2(t) &= t^3 - 2t^2 + t\\
    H_3(t) &= t^3 - t^2
\end{align*}

so that instead of having to manipulate polynomial coefficients
\[ f(t) = a\phi_3(t) + b\phi_2(t) + c\phi_1(t) + d\phi_0(t) \]

an easier point slope method can be used:
\[ f(t) = h_0H_0(t) + h_1H_1(t) + h_2H_2(t) + h_3H_3(t) \]

\includegraphics[scale=.5]{images/cubic-hermite-interpolation.png}
\begin{align*}
    h_0 &= P(0) = d\\
    h_1 &= P(1) = a + b + c + d\\
    h_2 &= P'(0) = c\\
    h_3 &= P'(1) = 3a + 2b + c
\end{align*}

Unknowns in this equation are $a$, $b$, $c$, and $d$, so a matrix can be used
to solve the systems of equations:
\[
    \begin{pmatrix}
        0 & 0 & 0 & 1\\
        1 & 1 & 1 & 1\\
        0 & 0 & 1 & 0\\
        3 & 2 & 1 & 0
    \end{pmatrix}
    \begin{pmatrix}
        a\\
        b\\
        c\\
        d
    \end{pmatrix}
    =
    \begin{pmatrix}
        h_0\\
        h_1\\
        h_2\\
        h_3
    \end{pmatrix}
\]

$a$, $b$, $c$, and $d$ can be obtained from $h$ values by inverting the matrix:
\[
    \begin{pmatrix}
        0 & 0 & 0 & 1\\
        1 & 1 & 1 & 1\\
        0 & 0 & 1 & 0\\
        3 & 2 & 1 & 0
    \end{pmatrix}^{-1}
    \begin{pmatrix}
        h_0\\
        h_1\\
        h_2\\
        h_3
    \end{pmatrix}
    =
    \begin{pmatrix}
        a\\
        b\\
        c\\
        d
    \end{pmatrix}
\]

From these solved $h$ values, $P(t)$ can now be converted to a form that is
easier for a user to manipulate, in terms of $h$ values:
\begin{align*}
    P(t) =& \ at^3 + bt^2 + ct + d\\
    =& \ (2h_0 - 2h_1 + h_2 + h_3)t^3\\
        &+ (-3h_0 + 3h_1 - 2h_2 - h_3)t^2\\
        &+ h_2t + h_0\\
    =& \ h_0(2t^3 - 3t^2 + 1) + h_1(-2t^3 + 3t^2) +\\
       & \ h_2(t^3 - 2t^2 + t) + h_3(t^3 - t^2)
\end{align*}

Each equation in $P(t) = h_0(2t^3 - 3t^2 + 1) +
h_1(-2t^3 + 3t^2) + h_2(t^3 - 2t^2 + t) + h_3(t^3 - t^2)$ that is multiplied
by an $h$ value is called a cubic hermite.

\subsubsection{More than 1D}

A parametric curve described by $\vec \gamma(t) = (\gamma_0(t),
\gamma_1(t))$ can be converted into hermite basis like this:

\includegraphics[scale=.5]{images/parametric-hermite.png}

where cubic hermite interpolation can be done for both dimensions.

\subsubsection{Cubic blossom}

The cubic blossom of a function $f(t)$ is $F(t_1,t_2,t_3)$.

Cubic blossoms have three properties:
\begin{enumerate}
    \item Symmetric
        \begin{itemize}
            \item $F(t_1,t_2,t_3) = F(t_1,t_3,t_2) = F(t_3,t_1,t_2) \cdots$
        \end{itemize}
    \item Affine
        \begin{itemize}
            \item $F(\alpha u + (1-\alpha)v,t_2,t_3) = \alpha F(u,t_2,t_3) + 
                 (1-\alpha)F(v,t_2,t_3)$
        \end{itemize}
    \item Diagonal
        \begin{itemize}
            \item $f(t) = F(t,t,t)$
        \end{itemize}
\end{enumerate}

Blossoming example: Given function $f(t) = t^3$, one way
to write $F(t_1,t_2,t_3)$ is $F(t_1,t_2,t_3) = t_1t_2t_3$.

\end{document}
